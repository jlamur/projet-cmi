\begin{tikzpicture}
    %%%%%%%%%%%%%%%%%%%%%%%%%%%%%
    %% ÉTATS DE JEU PRINCIPAUX %%
    %%%%%%%%%%%%%%%%%%%%%%%%%%%%%

    % Spécifications de l'état de jeu abstrait "niveau"
    % qui est une collection d'objets
    \umlabstract{Level}{
        name : string\\
        background : sprite\\
        music : string\\
        camera : vue\\
        gravity\_direction : direction\\
        total\_time : entier\\
        players : \texttt{[Player]}\\
        objects : \texttt{[Object]}\\
        zone : \texttt{[vecteur]}
    }{
        enable() : vide\\
        processEvent(event) : vide\\
        load(file : string) : vide\\
        save(file : string) : vide\\
        \umlvirt{frame() : vide}\\
        draw() : vide\\
        addObject(object : \texttt{Object})\\
        removeObject(object : \texttt{Object})
    }

    % Spécifications de l'état de jeu "éditeur" qui permet
    % de modifier des niveaux de jeu
    \umlclass[x=-6]{Editor}{
        selection : \texttt{[Object]}
    }{
        enable() : vide\\
        processEvent(event) : vide\\
        frame() : vide\\
        select(objet: \texttt{Object}) : vide\\
        selectAll() : vide\\
        clearSelection() : vide\\
    }
    \umlinherit{Level}{Editor}

    % Spécifications de l'état de jeu "jeu" qui permet
    % de jouer aux niveaux créés
    \umlclass[x=6]{Game}{
        mode : mode\\
        next\_frame\_time : temps\\
        pending\_kill : \texttt{[Object]}\\
        time\_left : flottant
    }{
        enable() : vide\\
        processEvent(event) : vide\\
        frame() : vide\\
        update() : vide\\
        kill(objet : \texttt{Object}) : vide
    }
    \umlinherit{Level}{Game}

    % Spécifications des objets du jeu
    \umlabstract[y=-6.5]{Object}{
        acceleration : vecteur\\
        velocity : vecteur\\
        position : vecteur\\
    }{
        \umlvirt{getForces(game : \texttt{Game}) : vecteur}\\
        \umlvirt{activate(game : \texttt{Game}, object : \texttt{Object}) : vecteur}\\
        \umlvirt{getAABB() : boîte}\\
        \umlvirt{getRadius() : flottant}\\
        \umlvirt{draw(level : \texttt{Level}) : vide}
    }
    \umlunicompo{Level}{Object}

    % Spécifications de l'objet "joueur"
    \umlclass[x=-4,y=-11]{Player}{
        player\_number : entier non-signé
    }{
        getForces(game : \texttt{Game}) : vecteur\\
        draw(level : \texttt{Level}) : vide\\
        activate(game : \texttt{Game}, object : \texttt{Object}) : vecteur\\
        getAABB() : boîte\\
        getRadius() : flottant\\
    }
    \umlinherit{Object}{Player}

    % Spécifications de l'objet "bloc"
    \umlclass[x=4,y=-11]{Block}{}{
        draw(level : \texttt{Level}) : vide\\
        prepareDraw(resources : \texttt{ResourceManager}) : vide\\
        activate(game : \texttt{Game}, object : \texttt{Object}) : vecteur\\
        getAABB() : boîte\\
        getRadius() : flottant\\
    }
    \umlinherit{Object}{Block}

    % Spécifications de l'objet "bloc de fin" permettant
    % quand il est activé de terminer le niveau
    \umlclass[x=-2,y=-14]{FinishBlock}{}{
        prepareDraw(resources : \texttt{ResourceManager}) : vide\\
        activate(game : \texttt{Game}, object : \texttt{Object}) : vecteur
    }
    \umlinherit{Block}{FinishBlock}

    % Spécifications de l'objet "bloc de gravité" permettant
    % quand il est activé de réorienter la gravité du niveau
    \umlclass[x=6,y=-14]{GravityBlock}{
        gravity\_direction : direction
    }{
        prepareDraw(resources : \texttt{ResourceManager}) : vide\\
        activate(game : \texttt{Game}, object : \texttt{Object}) : vecteur
    }
    \umlinherit{Block}{GravityBlock}

    % Spécifications de l'objet "bloc tueur" permettant
    % quand il est activé par un joueur de le tuer
    \umlclass[x=-2,y=-16.5]{KillBlock}{}{
        prepareDraw(resources : \texttt{ResourceManager}) : vide\\
        activate(game : \texttt{Game}, object : \texttt{Object}) : vecteur
    }
    \umlinherit{Block}{KillBlock}

    % Spécifications de l'objet "bloc d'échange" permettant
    % quand il est activé par un joueur d'échanger sa charge
    \umlclass[x=6,y=-16.5]{SwitchBlock}{}{
        prepareDraw(resources : \texttt{ResourceManager}) : vide\\
        activate(game : \texttt{Game}, object : \texttt{Object}) : vecteur
    }
    \umlinherit{Block}{SwitchBlock}
\end{tikzpicture}
