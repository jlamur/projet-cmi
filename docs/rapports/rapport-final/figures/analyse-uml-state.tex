\begin{tikzpicture}
    %%%%%%%%%%%%%%%%%%%%%%%
    %% CLASSES GÉNÉRALES %%
    %%%%%%%%%%%%%%%%%%%%%%%

    % Spécifications du gestionnaire de jeu
    \umlclass{Manager}{
        window : fenêtre\\
        resource\_manager : \texttt{ResourceManager}\\
        clock : horloge\\
        states : pile de \texttt{State}
    }{
        start() : vide\\
        pushState(state : \texttt{State}) : vide\\
        popState() : vide
    }

    % Spécifications du gestionnaire de ressources
    \umlclass[x=4,y=-4.5]{ResourceManager}{
        textures : dictionnaire \texttt{string -> texture}\\
        fonts : dictionnaire \texttt{string -> police}
        music : musique
    }{
        getTexture(name : string) : texture\\
        getFont(name : string) : texture\\
        getLevelPath(name : string) : string\\
        playMusic(name : string) : vide\\
        stopMusic() : vide
    }
    \umluniassoc{Manager}{ResourceManager}

    % Spécfications d'un état de jeu
    \umlabstract[x=-4,y=-4.5]{State}{
        manager : \texttt{Manager}
    }{
        \umlvirt{enable() : vide}\\
        \umlvirt{processEvent(event : événement) : vide}\\
        \umlvirt{frame() : vide}
    }
    \umlunicompo{Manager}{State}

    %%%%%%%%%%%%%%%%%%%%%%%%%%%%%
    %% ÉTATS DE JEU PRINCIPAUX %%
    %%%%%%%%%%%%%%%%%%%%%%%%%%%%%

    % Spécifications de l'état de jeu abstrait "niveau"
    % qui est une collection d'objets
    \umlabstract[x=7,y=-11]{Level}{}{
        \emph{Voir la figure suivante}
    }
    \umlinherit{State}{Level}

    %%%%%%%%%%%%%%%%%%%%%%%%%%
    %% ÉTATS DE JEU DU MENU %%
    %%%%%%%%%%%%%%%%%%%%%%%%%%

    % Spécifications de l'état de jeu "menu"
    % qui est le premier état au démarrage et permet
    % d'afficher les différents choix de jeu
    \umlclass[y=-11]{Menu}{
        background : sprite\\
        choices : \texttt{[string]}\\
        actions : \texttt{[callback]}\\
        selection : entier non-signé
    }{
        loadMainMenu() : vide\\
        loadLevelMenu() : vide\\
        loadEditorMenu() : vide\\
        launchGame(path : string) : vide\\
        launchEditor(path : string) : vide\\
        launchRules() : vide\\
        quit() : vide\\
        enable() : vide\\
        processEvent(event : événement) : vide\\
        frame() : vide
    }
    \umlinherit{State}{Menu}

    % Spécifications de l'état de jeu "règles" qui
    % affiche l'image décrivant les règles
    \umlclass[x=-7,y=-11]{Rules}{
        background : sprite\\
    }{
        processEvent(event : événement) : vide\\
        frame() : vide
    }
    \umlinherit{State}{Rules}
\end{tikzpicture}
