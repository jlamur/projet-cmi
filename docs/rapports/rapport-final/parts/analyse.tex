\chapter{Analyse du projet}

\section{Découpage du projet}

Le jeu est constitué d'une suite de niveaux organisés de manière
semblable à ceux d'un jeu de plateformes. Ces niveaux contiennent des entités.

Les entités du jeu sont multiples : blocs, blocs spéciaux, joueurs,
éléments de décor. Ces entités -- ou objets, interagissent entre elles
par un certain nombre de phénomènes physiques « naturels ». Pour répondre
à ce besoin, \textbf{un moteur physique} est nécessaire. Le moteur physique est
chargé de gérer les forces s'appliquant aux objets du jeu, de répondre
aux collisions entre objets et de faire évoluer les objets en conséquence
des forces qui leur sont appliquées.

Plusieurs moteurs physiques en 2D existent déjà dans le langage que nous
avons choisi, notamment Box2D. \cite{analyse-box2d}
Nous avons choisi d'implémenter le moteur
physique du jeu par nous-mêmes pour répondre aux besoins particuliers
(notamment la force d'attraction) et car cela nous permet de mettre en
pratique les savoirs acquis au premier semestre dans le module
de physique générale.

Les \textbf{niveaux du jeu} sont constitués de ces entités et d'autres
métadonnées. Pour pouvoir éditer les niveaux, les sauvegarder et
y rejouer plus tard, il est nécessaire de pouvoir les stocker
en dehors de la mémoire. Nous avons pour ce faire choisi de définir
un format de fichier binaire permettant leur stockage sur le disque.
Des fonctions pour coder et décoder ce format devront être écrites.

Skizzle propose différents \textbf{états de jeu}, notamment, on peut à tout moment
se trouver dans l'éditeur, dans le jeu en lui-même ou sur la vue des règles.
Pour pouvoir accéder à ces états, nous devons créer un menu. L'ensemble
des états du jeu doit être abstrait pour pouvoir être géré dans la classe
principale. Certains états du jeu proposeront des éléments interactifs
(boutons, barres d'outils, zones de texte) qui doivent être implémentés.

Enfin, les différents \textbf{objets du jeu} sont représentés à l'écran en
dessinant des textures. Nous avons également choisi d'ajouter des musiques
au jeu pour le rendre plus convivial. D'autres éléments graphiques doivent
être créés, par exemple le fond du menu. Tous ces éléments sont regroupés
dans l'univers graphique du jeu.

\section{Découpage du code}

Nous avons choisi d'organiser notre code selon le paradigme objet. La plupart
du code est sorti en dehors du \texttt{main}, dont la seule fonction est d'instancier
la classe \texttt{Manager} qui gère de manière abstraite le jeu et de démarrer
le premier état du jeu~: le menu.

\subsection{États, gestion des états et des ressources}

Un état du jeu modélise un écran pouvant être affiché. Une classe
abstraite \texttt{State} chapeaute toutes les classes d'états et permet
de requérir l'implémentation d'une interface commune~:

\begin{itemize}
    \item \texttt{enable()}~: cette méthode initialise l'état avant qu'il
    commence à être affiché. L'état implémentant cette méthode doit
    mettre en place les ressources globales utilisées comme la lecture
    de la musique, le titre de la fenêtre, les éléments de l'interface
    quand cette méthode est appelée~;

    \item \texttt{processEvent(event)}~: cette méthode est appelée avec
    un événement lorsque celui-ci est extrait par la SFML lors de la boucle
    principale. L'état est censé décider s'il souhaite traiter cet événement
    et, si oui, modifier ses variables en conséquence~;

    \item \texttt{frame()}~: cette méthode est appelée lorsque l'état
    doit dessiner une frame à l'écran. Pour éviter d'encombrer la boucle
    principale, l'état doit dessiner sa frame le plus rapidement possible.
\end{itemize}

Les états suivants sont implémentés et descendent de la classe \texttt{State}~:
\texttt{Rules} pour afficher les règles du jeu, \texttt{Menu} pour afficher
le menu du jeu, \texttt{Level} pour afficher les niveaux (soit l'éditeur,
soit le jeu en lui-même).

On définit \texttt{Manager} la classe qui gère les éléments principaux du jeu.
Notamment, \texttt{Manager} maintient une pile d'états qui est initialisée
contenant une seule instance de la classe \texttt{Menu} et peut être
empilée ou dépilée par les états. Par exemple, le menu peut empiler
un nouvel état instance de \texttt{Rules} pour « démarrer » la vue affichant
les règles. En tout temps, l'état en haut de la pile est celui qui est actif
(il reçoit les événements et est dessiné).

La librairie SFML permet de charger les ressources comme la musique,
les images et les polices. Cependant, recharger ces ressources à chaque
utilisation serait inefficace. La classe \texttt{ResourceManager} permet
de mutualiser ces ressources~: les états lui demandent les ressources
à obtenir et le gestionnaire de ressources s'arrange pour ne charger
la ressource qu'à la première demande et à la garder en mémoire par la suite.
Le gestionnaire des ressources mutualise l'accès aux polices, textures
et à la lecture de la musique.

La figure \ref{fig:analyse-uml-state} résume les classes de gestion
d'états et de ressources présentées.

\newgeometry{left=1cm,top=2cm,bottom=2cm,right=1cm}
\begin{figure}[p!]
    \centering
    \begin{tikzpicture}
    %%%%%%%%%%%%%%%%%%%%%%%
    %% CLASSES GÉNÉRALES %%
    %%%%%%%%%%%%%%%%%%%%%%%

    % Spécifications du gestionnaire de jeu
    \umlclass{Manager}{
        window : fenêtre\\
        resource\_manager : \texttt{ResourceManager}\\
        clock : horloge\\
        states : pile de \texttt{State}
    }{
        start() : vide\\
        pushState(state : \texttt{State}) : vide\\
        popState() : vide
    }

    % Spécifications du gestionnaire de ressources
    \umlclass[x=4,y=-4.5]{ResourceManager}{
        textures : dictionnaire \texttt{string -> texture}\\
        fonts : dictionnaire \texttt{string -> police}
        music : musique
    }{
        getTexture(name : string) : texture\\
        getFont(name : string) : texture\\
        getLevelPath(name : string) : string\\
        playMusic(name : string) : vide\\
        stopMusic() : vide
    }
    \umluniassoc{Manager}{ResourceManager}

    % Spécfications d'un état de jeu
    \umlabstract[x=-4,y=-4.5]{State}{
        manager : \texttt{Manager}
    }{
        \umlvirt{enable() : vide}\\
        \umlvirt{processEvent(event : événement) : vide}\\
        \umlvirt{frame() : vide}
    }
    \umlunicompo{Manager}{State}

    %%%%%%%%%%%%%%%%%%%%%%%%%%%%%
    %% ÉTATS DE JEU PRINCIPAUX %%
    %%%%%%%%%%%%%%%%%%%%%%%%%%%%%

    % Spécifications de l'état de jeu abstrait "niveau"
    % qui est une collection d'objets
    \umlabstract[x=7,y=-11]{Level}{}{
        \emph{Voir la figure suivante}
    }
    \umlinherit{State}{Level}

    %%%%%%%%%%%%%%%%%%%%%%%%%%
    %% ÉTATS DE JEU DU MENU %%
    %%%%%%%%%%%%%%%%%%%%%%%%%%

    % Spécifications de l'état de jeu "menu"
    % qui est le premier état au démarrage et permet
    % d'afficher les différents choix de jeu
    \umlclass[y=-11]{Menu}{
        background : sprite\\
        choices : \texttt{[string]}\\
        actions : \texttt{[callback]}\\
        selection : entier non-signé
    }{
        loadMainMenu() : vide\\
        loadLevelMenu() : vide\\
        loadEditorMenu() : vide\\
        launchGame(path : string) : vide\\
        launchEditor(path : string) : vide\\
        launchRules() : vide\\
        quit() : vide\\
        enable() : vide\\
        processEvent(event : événement) : vide\\
        frame() : vide
    }
    \umlinherit{State}{Menu}

    % Spécifications de l'état de jeu "règles" qui
    % affiche l'image décrivant les règles
    \umlclass[x=-7,y=-11]{Rules}{
        background : sprite\\
    }{
        processEvent(event : événement) : vide\\
        frame() : vide
    }
    \umlinherit{State}{Rules}
\end{tikzpicture}

    \caption{Gestion des états et des ressources dans le jeu}
    \label{fig:analyse-uml-state}
\end{figure}
\restoregeometry

\subsection{Niveau et objets}

La classe \texttt{Level} définit les niveaux, qui sont des collections
d'objets. Elle définit la méthode pour dessiner tous les objets d'un niveau,
le charger, le sauvegarder dans un fichier, ajouter ou supprimer des objets.
Elle ne définit pas la méthode \texttt{frame()} que tous les états doivent
implémenter, elle n'est donc pas un état en tant que tel.

Deux classes dérivent de \texttt{Level} : \texttt{Game} pour jouer aux
niveaux et \texttt{Editor} pour les éditer. L'abstraction en
\texttt{Level} permet d'éviter la duplication de code notamment en
ce qui concerne la gestion des objets contenus.

Les classes de niveaux manipulent des collections d'objets. Les objets
modélisent toutes les entités du jeu : les joueurs, les blocs et les
blocs spéciaux. Une classe abstrait les fonctionnalités de tous les objets,
\texttt{Object}.

Les classes \texttt{Block}, définissant l'apparence et le comportement
des blocs, et \texttt{Player}, définissant l'apparence et le comportement
des joueurs, descendent directement d'\texttt{Object}. Enfin, on définit
des blocs spéciaux, qui peuvent réaliser des actions particulières~:

\begin{itemize}
    \item le bloc de gravité modifie la direction de la gravité
    dans un niveau lorsqu'une entité entre en contact avec lui.
    Il ne peut être activé qu'une seule fois par partie~;

    \item le bloc changeur échange la polarité de l'entité
    entrant en contact avec lui. Il ne peut être activé qu'une seule
    fois par partie~;

    \item le bloc tueur tue le joueur entrant en contact avec lui
    et fait perdre la partie (le niveau passe en mode « perdu »)~;

    \item le bloc d'arrivée tue le joueur entrant en contact et lorsqu'il
    ne reste plus de joueurs fait gagner la partie (le niveau passe
    en mode « gagné »).
\end{itemize}

La figure \ref{fig:analyse-uml-level} résume les classes de niveaux
et d'objets.

\newgeometry{left=1cm,top=2cm,bottom=2cm,right=1cm}
\begin{figure}[p!]
    \centering
    \begin{tikzpicture}
    %%%%%%%%%%%%%%%%%%%%%%%%%%%%%
    %% ÉTATS DE JEU PRINCIPAUX %%
    %%%%%%%%%%%%%%%%%%%%%%%%%%%%%

    % Spécifications de l'état de jeu abstrait "niveau"
    % qui est une collection d'objets
    \umlabstract{Level}{
        name : string\\
        background : sprite\\
        music : string\\
        camera : vue\\
        gravity\_direction : direction\\
        total\_time : entier\\
        players : \texttt{[Player]}\\
        objects : \texttt{[Object]}\\
        zone : \texttt{[vecteur]}
    }{
        enable() : vide\\
        processEvent(event) : vide\\
        load(file : string) : vide\\
        save(file : string) : vide\\
        \umlvirt{frame() : vide}\\
        draw() : vide\\
        addObject(object : \texttt{Object})\\
        removeObject(object : \texttt{Object})
    }

    % Spécifications de l'état de jeu "éditeur" qui permet
    % de modifier des niveaux de jeu
    \umlclass[x=-6]{Editor}{
        selection : \texttt{[Object]}
    }{
        enable() : vide\\
        processEvent(event) : vide\\
        frame() : vide\\
        select(objet: \texttt{Object}) : vide\\
        selectAll() : vide\\
        clearSelection() : vide\\
    }
    \umlinherit{Level}{Editor}

    % Spécifications de l'état de jeu "jeu" qui permet
    % de jouer aux niveaux créés
    \umlclass[x=6]{Game}{
        mode : mode\\
        next\_frame\_time : temps\\
        pending\_kill : \texttt{[Object]}\\
        time\_left : flottant
    }{
        enable() : vide\\
        processEvent(event) : vide\\
        frame() : vide\\
        update() : vide\\
        kill(objet : \texttt{Object}) : vide
    }
    \umlinherit{Level}{Game}

    % Spécifications des objets du jeu
    \umlabstract[y=-6.5]{Object}{
        acceleration : vecteur\\
        velocity : vecteur\\
        position : vecteur\\
    }{
        \umlvirt{getForces(game : \texttt{Game}) : vecteur}\\
        \umlvirt{activate(game : \texttt{Game}, object : \texttt{Object}) : vecteur}\\
        \umlvirt{getAABB() : boîte}\\
        \umlvirt{getRadius() : flottant}\\
        \umlvirt{draw(level : \texttt{Level}) : vide}
    }
    \umlunicompo{Level}{Object}

    % Spécifications de l'objet "joueur"
    \umlclass[x=-4,y=-11]{Player}{
        player\_number : entier non-signé
    }{
        getForces(game : \texttt{Game}) : vecteur\\
        draw(level : \texttt{Level}) : vide\\
        activate(game : \texttt{Game}, object : \texttt{Object}) : vecteur\\
        getAABB() : boîte\\
        getRadius() : flottant\\
    }
    \umlinherit{Object}{Player}

    % Spécifications de l'objet "bloc"
    \umlclass[x=4,y=-11]{Block}{}{
        draw(level : \texttt{Level}) : vide\\
        prepareDraw(resources : \texttt{ResourceManager}) : vide\\
        activate(game : \texttt{Game}, object : \texttt{Object}) : vecteur\\
        getAABB() : boîte\\
        getRadius() : flottant\\
    }
    \umlinherit{Object}{Block}

    % Spécifications de l'objet "bloc de fin" permettant
    % quand il est activé de terminer le niveau
    \umlclass[x=-2,y=-14]{FinishBlock}{}{
        prepareDraw(resources : \texttt{ResourceManager}) : vide\\
        activate(game : \texttt{Game}, object : \texttt{Object}) : vecteur
    }
    \umlinherit{Block}{FinishBlock}

    % Spécifications de l'objet "bloc de gravité" permettant
    % quand il est activé de réorienter la gravité du niveau
    \umlclass[x=6,y=-14]{GravityBlock}{
        gravity\_direction : direction
    }{
        prepareDraw(resources : \texttt{ResourceManager}) : vide\\
        activate(game : \texttt{Game}, object : \texttt{Object}) : vecteur
    }
    \umlinherit{Block}{GravityBlock}

    % Spécifications de l'objet "bloc tueur" permettant
    % quand il est activé par un joueur de le tuer
    \umlclass[x=-2,y=-16.5]{KillBlock}{}{
        prepareDraw(resources : \texttt{ResourceManager}) : vide\\
        activate(game : \texttt{Game}, object : \texttt{Object}) : vecteur
    }
    \umlinherit{Block}{KillBlock}

    % Spécifications de l'objet "bloc d'échange" permettant
    % quand il est activé par un joueur d'échanger sa charge
    \umlclass[x=6,y=-16.5]{SwitchBlock}{}{
        prepareDraw(resources : \texttt{ResourceManager}) : vide\\
        activate(game : \texttt{Game}, object : \texttt{Object}) : vecteur
    }
    \umlinherit{Block}{SwitchBlock}
\end{tikzpicture}

    \caption{Classes du niveau}
    \label{fig:analyse-uml-level}
\end{figure}
\restoregeometry
