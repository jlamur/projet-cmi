\chapter{Conclusion}

\section{Perspectives}
Par manque de temps, de nombreuses perspectives d'amélioration n'ont pas pu
aboutir. Nous aurions pu améliorer l'interface graphique (ce qui est prévu
avant l'oral), faire plus de niveaux et donc ajouter des musiques pour ces
nouveaux niveaux, ou encore ajouter les menus pause, victoire et défaite.

\section{Conclusions}

\subsection{Fonctionnement de l'application}

Nous avons remarqué qu'un message d'erreur « l'application ne répond pas »
s'affiche dans certaines circonstances que nous n'avons pas réussi à isoler,
bien que le programme réponde toujours.

Lors de la rotation de la caméra à la suite d'un changement dans la direction
de la gravité, la caméra ne réalise pas toujours le mouvement le plus rapide
et peut effectuer une rotation à 270\degree au lieu de 90\degree dans certaines
circonstances.

Dans certains cas les blocs déplaçables, si placés en trop grandes quantités,
se traversent. De plus, l'algorithme de collisions étant discret, un objet
ayant une vitesse trop élevée peut passer à travers un autre.

Enfin, la force d'attraction ne réagit pas tout à fait comme nous l'avions
imaginée initialement : elle est moins souple. Cela est probablement lié
à un mauvais ajustement dans les constantes du moteur physique. Cette
force fonctionne malgré tout.

Cependant, les menus ainsi que les retours aux menus fonctionnent bien,
l'affichage est fluide, l'éditeur est fonctionnel et comporte de nombreuses
fonctionnalités intuitives. Les ressources ne posent pas de problème,
les textures se chargent et les musiques se déclenchent correctement
et bouclent comme prévu.

Globalement, le jeu est fonctionnel et ne pose problème que dans des
cas très particuliers, il reste tout de même stable malgré le temps
limité que nous avions pour régler les problèmes.

\subsection{Fonctionnement du groupe de travail}

Le principal problème du groupe fut le sens des priorités. Nous nous sommes
attardés sur des détails avant que le tout soit fonctionnel, ce qui a
perturbé notre gestion du temps. Malgré cela, nous avons réussi à nous
organiser pour gagner en efficacité en partageant les tâches les plus
chronophages.

Le fait de refaire fréquemment le diagramme de Gantt nous a permis de
rattraper en partie le retard que nous avions pris. Au niveau de la
communication, nous n'avons eu aucun problème car nous nous
retrouvions très souvent sur Skype ainsi qu'à l'université afin
de discuter de tout problème et optimiser nos chances de trouver
rapidement une solution.

Malgré une gestion du temps perturbée, notre organisation et notre
communication nous ont permis de compenser les défauts du groupe,
et ainsi d'avancer à un rythme convenable.

Ainsi, grâce à notre détermination et notre coopération, nous avons
su franchir les difficultés et ainsi proposer un jeu fonctionnel et amusant.
