\chapter{Introduction}

\section {Généralités}

Dans le cadre du module Projet C.M.I du second semestre de licence 1,
nous avons développé en équipe un jeu vidéo nommé « Skizzle ».
Notre groupe est composé de trois personnes, Mattéo Delabre, Maëlle Beuret
et Rémi Cérès.

L'objectif principal de ce projet est la création d'un jeu vidéo fonctionnel. Le jeu utilise les bases d'algorithmique apprises au premier semestre ainsi que nos connaissances et
recherches personnelles.\\
La création de ce jeu nous a permis d'acquérir des notions de gestion de projet, ainsi que le travail en collaboration et l'apprentissage de la
communication au sein d'une équipe.
\\

Le développement du projet s'est déroulé sur une période d'un mois et une semaine,
soit du vendredi 4 mars 2016 au lundi 11 avril 2016 inclus.\\
Chaque vendredi,
lors de la séance de trois heures consacrée au projet C.M.I, un point était réalisé avec chaque membre de l'équipe.
Une répartition des tâches et des objectifs
à atteindre pour la semaine suivante étaient déterminés. De plus, durant la semaine nous avons communiqué afin de faciliter l'organisation, de nous coordonner et de nous entraider.\\


\section {Sujet}
Le projet a pour sujet la réalisation d'un jeu vidéo
fonctionnel en groupe ou en individuel.\\
Le choix du jeu et du langage de programmation est libre; cela peut être une création ou bien un jeu existant.\\

Notre choix s'est porté sur un jeu avec un principe original inspiré des jeux de
plates-formes, de coopération et de réflexion. Nous l'avons appelé "Skizzle".\\

Il s’agit pour deux joueurs de faire traverser deux
balles à travers un niveau.
La conception du jeu s’appuie sur des éléments physiques pour proposer des
niveaux sous forme de casse-têtes, dans lesquels les deux participants doivent
s’entraider et réfléchir pour parvenir à la fin.

%Skizzle est constitué d'une suite de niveaux dont le but est de tous les
%terminer. Chaque niveaux est composé de bloc pouvant avoir des interaction avec
%les balle ou sur la physique du jeu, comme par exemple modifié le sens de
%la gravité.\\
%Le ou les deux joueurs contrôle chacun une balle qu'ils peuvent uniquement
%déplacer vers la droite et la gauche de l'écran. Tous les élément du jeux,
%c'est a dire les blocs et les balles peuvent être aimantés.\\
%
%Dans chaque niveaux le ou les joueurs doivent réfléchir et d’entraidé en
%s’appuyant sur les différentes interaction possible avec la physique du jeu
%pour parvenir a la fin.
\section{Cahier des charges}


\subsection{Généralités}

\begin{itemize}
    \item \textbf{Fonctionnalités}
    \begin{itemize}
        \item Divertir et faire réfléchir les utilisateurs.
        \item Multi-plateformes (Linux, OS X, Windows).
    \end{itemize}

    \item \textbf{Contraintes}
    \begin{itemize}
        \item Interface pas adaptée au tactile.
        \item Présence de deux joueurs.
    \end{itemize}
\end{itemize}

\subsection{Menu}

\begin{itemize}
    \item Permet à l'utilisateur de choisir son mode de jeu (éditer/jouer).
    \item Permet d'afficher la liste des niveaux de manière dynamique et automatique.
    \item Possibilité d'utiliser aussi bien la souris que le clavier.
\end{itemize}

\subsection{Moteur physique}

\begin{itemize}
    \item Gestion de la force d'attraction.
    \item Gestion de la force de gravité.
    \item Gestion des forces de frottement.
    \item Gestion des collisions.
\end{itemize}

\subsection{Éditeur}

\begin{itemize}
    \item Permet de créer un niveau.
    \item Permet de modifier un niveau existant.
    \item Permet de placer tous les types d'objets.
    \item Permet de placer la position initiale des joueurs.
    \item Permet de modifier la polarité des objets.
    \item Permet de sauvegarder un niveau modifié ou créé.
    \item Permet de tester le niveau créé directement dans l'éditeur.
    \item Permet de modifier la taille et la forme de la zone jouable.
\end{itemize}
