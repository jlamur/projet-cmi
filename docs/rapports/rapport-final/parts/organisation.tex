\chapter{Organisation du projet}

\section{Organisation du travail}

Lors de la réunion initiale les tâches ont été réparties selon les capacités
de chacun. En dehors des réunions, chaque membre du groupe a travaillé en
autonomie, en communiquant son avancement au reste du groupe.

Les réunions suivantes étaient consacrées au passage en revue des modifications
effectuées par chacun, à l'analyse des tâches restant à effectuer et à leur
distribution entre les membres du groupe, toujours selon leurs capacités
et leur disponibilité.

La répartition finale des tâches telles qu'elles sont été accomplies
est présentée dans la figure \ref{fig:organisation-gantt}.

Certaines parties du développement nécessitaient plusieurs personnes et
étaient ainsi partagées entre certains membres du groupe. La conception
et les tests des niveaux, le fond des menus ainsi que les décors furent
réalisés par Rémi et Maëlle, et la gestion de projet par tout le groupe.

\newgeometry{left=2cm,top=1.5cm,bottom=1.5cm,right=2cm}
\begin{figure}[p!]
    \centering
    \begin{ganttchart}[
    x unit=.4cm,
    y unit title=0.5cm,
    title height=1,
    y unit chart=.6cm,
    vgrid=true
]{0}{32}
    % raccourci pour changer le style des barres
    \newcommand{\ganttbstyle}[1]{\ganttset{bar/.append style={fill=#1}}}

    \gantttitle{Développement du projet}{20}
    \gantttitle{Rapport}{4}
    \gantttitle{Préparation soutenance}{9}\\
    \gantttitle{04/03}{4}
    \gantttitle{11/03}{4}
    \gantttitle{18/03}{4}
    \gantttitle{25/03}{4}
    \gantttitle{01/04}{4}
    \gantttitle{09/04}{4}
    \gantttitle{16/04}{4}
    \gantttitle{23/04}{5}\\

    \ganttgroup{Moteur physique}{0}{14}\\
    \ganttbstyle{matred}
    \ganttbar{Organisation des classes}{0}{6}\\
    \ganttbstyle{matgreen}
    \ganttbar{Gestion de l'affichage}{0}{8}\\
    \ganttbstyle{matred}
    \ganttbar{Implémentation des forces}{0}{8}\\
    \ganttbar{Gestion des collisions}{4}{14}\\
    \ganttbstyle{matblue}
    \ganttbar{Tests du moteur}{4}{14}\\

    \ganttgroup{Niveaux du jeu}{6}{32}\\
    \ganttbstyle{matblue}
    \ganttbar{Définition du format de fichier}{6}{8}\\
    \ganttbstyle{matred}
    \ganttbar{Lecture et écriture du format}{12}{16}\\
    \ganttbar{Création de l'éditeur}{16}{21}\\
    \ganttbar{Blocs spécialisés}{18}{21}\\
    \ganttbstyle{matblue}
    \ganttbar{Niveau de test}{19}{21}\\
    \ganttbstyle{matgray}
    \ganttbar{Conception des niveaux}{19}{32}\\
    \ganttbar{Tests des niveaux}{19}{32}\\

    \ganttgroup{Interface du jeu}{10}{32}\\
    \ganttbstyle{matred}
    \ganttbar{Découpage en états de jeu}{10}{14}\\
    \ganttbstyle{matgreen}
    \ganttbar{Création du menu}{12}{18}\\
    \ganttbstyle{matred}
    \ganttbar{Interface de l'éditeur}{16}{21}\\
    \ganttbar{Menus gagné, perdu, pause}{24}{32}\\

    \ganttgroup{Univers graphique}{8}{28}\\
    \ganttbstyle{matgreen}
    \ganttbar{Création des musiques}{8}{16}\\
    \ganttbstyle{matgray}
    \ganttbar{Fond du menu}{12}{20}\\
    \ganttbar{Décor}{14}{21}\\
    \ganttbstyle{matblue}
    \ganttbar{Textures blocs}{18}{21}\\
    \ganttbstyle{matred}
    \ganttbar{Implémentation décor}{24}{28}\\

    \ganttgroup{Gestion de projet}{21}{32}\\
    \ganttbstyle{matgray}
    \ganttbar{Rédaction du rapport}{21}{25}\\
    \ganttbar{Tests pré-présentation}{25}{32}\\
    \ganttmilestone{Présentation du jeu}{32}
\end{ganttchart}

    \caption{
        Diagramme de la répartition des tâches. En vert, les tâches
        affectées à Maëlle~; en bleu, les tâches affectées à Rémi~;
        en rouge, les tâches affectées à Mattéo~; en gris, les
        tâches résolues en groupe
    }
    \label{fig:organisation-gantt}
\end{figure}
\restoregeometry

\section{Outils de développement}

Tout le code du programme est centralisé sur un dépôt GitHub et les membres
du groupe utilisent Git pour synchroniser leur travail, travailler
indépendamment, vérifier le travail des autres ou récupérer
d'anciennes versions.
\cite{organisation-github-skizzle}

Pour le développement du jeu, nous avons choisi le C++ car il s'agit du
langage que nous apprenons cette année~: d'une part, nous possédons
donc des acquis avec de langage, et d'autre part ce projet nous permet
de fixer nos connaissances du C++. Une autre raison est la quantité
importante de documentation sur le web, de nombreuses bibliothèques
et d'une communauté importante.

Nous avons fait le choix de la SFML comme librairie graphique et d'utiliser
la spécification C++11 qui apporte des fonctionnalités non-négligeables
(notamment concernant la gestion de la mémoire avec les pointeurs intelligents).
\cite{organisation-smartptr}

Pour écrire le code, nous avons utilisé différents éditeurs de texte
(atom et gedit). Nous compilons notre programme avec le compilateur g++.
Pour faciliter la compilation, nous avons utilisé CMake.
