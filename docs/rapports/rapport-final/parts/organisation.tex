\chapter{Organisation du projet}

\section{Organisation du travail}

Chaque membre du groupe a travaillé en autonomie. Les réunions lors des séances
prévues étaient consacrées aux explications sur le travail de chacun ainsi qu'à
la répartition des tâches pour la semaine suivante.

Chacun d'entre nous était chargé de tâches spécifiques (voir figure 1).
Certaines parties du développement nécessitaient plusieurs personne et
étaient ainsi partagées entre certains membres du groupe. La conception
et les tests des niveaux, le fond des menus ainsi que les décors furent
réalisés par Rémi et Maëlle, et la gestion de projet par tout le groupe.

Tout au long de la réalisation du projet, nous communiquions par Skype
afin de s'informer de l'avancement et réfléchir à des solutions lorsqu'un
problème était rencontré.

\begin{landscape}
\begin{figure}[p!]
    \scalebox{0.56}{\begin{gantt}{32}{32}
    \begin{ganttitle}
        \titleelement{Semaine 1 : 04/03 - 10/03}{4}
        \titleelement{Semaine 2 : 11/03 - 17/03}{4}
        \titleelement{Semaine 3 : 18/03 - 24/03}{4}
        \titleelement{Semaine 4 : 25/03 - 31/03}{4}
        \titleelement{Semaine 5 : 01/04 - 08/04}{4}
        \titleelement{Semaine 6 : 09/04 - 15/04}{4}
        \titleelement{Semaine 7 : 16/04 - 22/04}{4}
        \titleelement{Semaine 8 : 23/04 - 29/04}{4}
    \end{ganttitle}

    \ganttgroup{Moteur physique}{0}{14}
    \ganttbar[color=red]{Organisation des classes}{0}{6}
    \ganttbar[color=green]{Gestion de l'affichage}{0}{8}
    \ganttbar[color=red]{Implémentation des forces}{0}{8}
    \ganttbar[color=red]{Gestion des collisions}{4}{10}
    \ganttbar[color=cyan]{Tests du moteur}{4}{10}

    \ganttgroup{Niveaux du jeu}{6}{26}
    \ganttbar[color=cyan]{Définition du format de fichier}{6}{2}
    \ganttbar[color=red]{Lecture et écriture du format}{12}{4}
    \ganttbar[color=red]{Création de l'éditeur}{16}{5}
    \ganttbar[color=red]{Blocs spécialisés}{18}{3}
    \ganttbar[color=cyan]{Niveau de test}{19}{2}
    \ganttbar{Conception des niveaux}{19}{13}
    \ganttbar{Tests des niveaux}{19}{13}

    \ganttgroup{Interface du jeu}{10}{22}
    \ganttbar[color=red]{Découpage en états de jeu}{10}{4}
    \ganttbar[color=green]{Création du menu}{12}{6}
    \ganttbar[color=red]{Interface de l'éditeur}{16}{5}
    \ganttbar[color=red]{Menus gagné, perdu, pause}{24}{8}

    \ganttgroup{Univers graphique}{8}{20}
    \ganttbar[color=green]{Création des musiques}{8}{8}
    \ganttbar{Fond du menu}{12}{8}
    \ganttbar{Décor}{14}{7}
    \ganttbar[color=cyan]{Textures blocs}{18}{3}
    \ganttbar[color=red]{Implémentation décor}{24}{4}

    \ganttgroup{Gestion de projet}{21}{11}
    \ganttmilestone{Rendu du code}{21}
    \ganttbar{Rédaction du rapport}{21}{4}
    \ganttmilestone{Rendu du rapport}{25}
    \ganttbar{Tests pré-présentation}{25}{7}
    \ganttmilestone{Présentation du jeu}{32}
\end{gantt}
}
    \caption{
        Diagramme de la répartition des tâches. En vert, les tâches
        affectées à Maëlle~; en cyan, les tâches affectées à Rémi~;
        en rouge, les tâches affectées à Mattéo~; en noir, les
        tâches résolues en groupe
    }
    \label{fig:organisation-gantt}
\end{figure}
\end{landscape}

\section{Outils de développement}

Nous avons choisi le C++ tout d'abord car il s'agit du langage que nous
apprenons cette année. De plus, il possède de nombreuses bibliothèques.
C'est un langage très utilisé dont le code est élégant. Parmi les bibliothèques
graphiques, nous avons choisi la SFML car son utilisation est simple et
elle correspondait bien à nos besoins.

Pour écrire le code, nous avons utilisé différents éditeurs de texte
(atom et gedit). Nous compilions notre programme avec g++. Pour faciliter
la compilation, nous avons utilisé CMake.

Git nous a permis de gérer les versions du programme, avec le gestionnaire
de projet GitHub, où nous déposions le code ainsi que les documents tels
que le diagramme UML de classes permettant de s'y retrouver plus facilement
dans les nombreuses classes que nous avons créées.
