\chapter*{Étude comparative}
\addcontentsline{toc}{chapter}{Étude comparative}

Le jeu de la vie permet, avec des règles très simples, de produire
des motifs extrêmement complexes. Ces motifs peuvent nécessiter de
nombreuses générations sur de grandes surfaces, d'où la nécessité
d'un algorithme évitant les calculs inutiles. En effet, on peut
remarquer que beaucoup de motifs se répètent au cours des générations
et qu'il n'est pas utile de calculer l'état de zones isolées
des cellules vivantes. C'est l'objectif de l'algorithme \emph{Hashlife}.

D'un point de vue algorithmique, le jeu de go est également
intéressant. Le décompte des points demande de pouvoir identifier
les territoires des joueurs, ce à quoi un algorithme de remplissage par
diffusion se prête bien, comme montré au cours de l'étude algorithmique.
Cependant, le décompte peut aussi demander (en fonction
des règles que l'on applique) de retirer les pierres « mortes »
c'est à dire celles qui n'auraient pas pu être sauvées
de la prise même en contiuant de jouer. On ne peut pas imaginer pouvoir
concevoir un algorithme qui détecte de telles pierres sans avoir
une intelligence artificielle assez performante pour pouvoir simuler
les coups suivants. Les intelligences artificielles en matière
de jeu de go restent peu performantes, car les techniques
classiques de recherche de solutions employées notamment avec
les échecs ne sont pas envisageables, au vu du nombre de possibilités.

Le jeu de plateformes présente un intérêt différent. D'une part,
la quantité de travail à fournir est largement supérieure à celle des autres
jeux présentés, et nécessite donc la coordination d'une équipe.
D'autre part, les algorithmes à concevoir sont directement
à mettre en relation avec l'enseignement de mécanique classique du
premier semestre, ce qui permet de le valoriser. Enfin, un tel jeu
a un résultat visuel plus à même d'être présenté en vidéo ou
devant un public, puisqu'il est plus dynamique que le jeu de la
vie ou le jeu de go.

Chaque jeu présenté dans ce rapport demande une organisation différente car le
volume de travail requis va en croissant. En effet, alors que le jeu de la vie
peut être développé par une seule personne en 40 heures totales, le jeu de
plateformes demande au moins 175 heures et donc la collaboration d'au moins
3 personnes.
