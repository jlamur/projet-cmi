%%%%%%%%%%
% Macros %
%%%%%%%%%%

% Associe les couleurs en fonction de l'état d'une cellule
\newcommand{\cellcolor}[1]{%
\ifnum\pdfstrcmp{#1}{alive}=0 matblue\fi%
\ifnum\pdfstrcmp{#1}{dying}=0 matred\fi%
\ifnum\pdfstrcmp{#1}{nascent}=0 matgreen\fi}

% Dessine une cellule en X, Y dans l'état donné (doit être
% utilisé dans un environnement TikZ)
\newcommand{\fillcell}[3]{
\ifnum\pdfstrcmp{#3}{stillborn}=0
\draw [fill=\cellcolor{dying}] (#1,#2) -- (#1+1,#2) -- (#1+1,#2+1) -- (#1,#2);
\draw [fill=\cellcolor{nascent}] (#1,#2) -- (#1,#2+1) -- (#1+1,#2+1) -- (#1,#2);
\else
\draw [fill=\cellcolor{#3}] (#1,#2) rectangle (#1+1,#2+1);
\fi
}

%%%%%%%%%%
% Source %
%%%%%%%%%%

\chapter{Jeu de la vie}

Le jeu de la vie, inventé par John Conway en 1970, est l'exemple le plus
populaire d'automate cellulaire.
\cite{gof-wikipedia}
Un automate cellulaire est un ensemble
de cellules pouvant être dans plusieurs états différents et dont l'état
suivant est entièrement déterminé à partir de l'état actuel.
\cite{gof-automata}

Il présente un intérêt théorique car il a été démontré que le jeu de la vie
est Turing-complet, c'est-à-dire que n'importe quel algorithme peut
y être implémenté.

\section{Principes}\label{sec:principes}

Le jeu de la vie est composé d'une grille infinie de cellules pouvant
être soit mortes soit vivantes. Le jeu est constitué d'états : un état du jeu
est l'ensemble des états (vivante ou morte) de chacune de ses cellules à
un moment précis. Seul l'état initial est fourni, les états suivants sont calculés
de proche en proche selon les règles suivantes~:

\begin{enumerate}
    \item si une cellule a trois voisines à l'état $n$, elle est vivante à l'état $n+1$ ;
    \item si une cellule a deux voisines à l'état $n$, elle persévère dans son état ;
    \item si une cellule a moins de deux ou plus de trois voisines à l'état $n$, elle est morte à l'état $n+1$.
    \cite{gof-conwaylife}
\end{enumerate}

Les figures~\ref{fig:life-ex-1}, \ref{fig:life-ex-2} et \ref{fig:life-ex-3} montrent
trois situations typiques du jeu de la vie sur des portions de grilles.
Dans ces figures, une case \tikz[scale=.2]{\fillcell{0}{0}{nascent}} représente
une cellule qui vient de naître, une case \tikz[scale=.2]{\fillcell{0}{0}{dying}}
représente une cellule qui va mourir à l'état suivant et une case
\tikz[scale=.2]{\fillcell{0}{0}{alive}} représente une cellule vivante.

\begin{figure}[p!]
    \centering
    \begin{tikzpicture}[scale=.5]
    \begin{scope}
        \draw (0,0) grid (5, 5);
        \draw[thick, scale=5] (0, 0) grid (1, 1);

        \fillcell{1}{2}{dying}
        \fillcell{2}{1}{alive}
        \fillcell{3}{1}{dying}

        \node[anchor=center] at (2.5,-0.5) {$n=0$};
    \end{scope}

    \begin{scope}[xshift=6cm]
        \draw (0,0) grid (5, 5);
        \draw[thick, scale=5] (0, 0) grid (1, 1);

        \fillcell{2}{1}{dying}
        \fillcell{2}{2}{stillborn}

        \node[anchor=center] at (2.5,-0.5) {$n=1$};
    \end{scope}

    \begin{scope}[xshift=12cm]
        \draw (0,0) grid (5, 5);
        \draw[thick, scale=5] (0, 0) grid (1, 1);

        \node[anchor=center] at (2.5,-0.5) {$n=2$};
    \end{scope}
\end{tikzpicture}

    \caption{Mort par sous-population}
    \label{fig:life-ex-1}
\end{figure}

\begin{figure}[p!]
    \centering
    \begin{tikzpicture}[scale=.5]
    \begin{scope}
        \draw (0,0) grid (5, 5);
        \draw[thick, scale=5] (0, 0) grid (1, 1);

        \fillcell{2}{1}{stillborn}
        \fillcell{2}{2}{alive}
        \fillcell{2}{3}{stillborn}

        \node[anchor=center] at (2.5,-0.5) {$n=2k$};
    \end{scope}

    \begin{scope}[xshift=6cm]
        \draw (0,0) grid (5, 5);
        \draw[thick, scale=5] (0, 0) grid (1, 1);

        \fillcell{1}{2}{stillborn}
        \fillcell{2}{2}{alive}
        \fillcell{3}{2}{stillborn}

        \node[anchor=center] at (2.5,-0.5) {$n=2k+1$};
    \end{scope}
\end{tikzpicture}

    \caption{Un oscillateur, une configuration qui se répète indéfiniment}
    \label{fig:life-ex-2}
\end{figure}

\begin{figure}[p!]
    \centering
    \begin{tikzpicture}[scale=.5]
    \begin{scope}
        \draw (0,0) grid (5, 5);
        \draw[thick, scale=5] (0, 0) grid (1, 1);

        \fillcell{2}{2}{alive}
        \fillcell{3}{2}{alive}
        \fillcell{4}{2}{dying}
        \fillcell{2}{3}{alive}
        \fillcell{2}{4}{dying}

        \node[anchor=center] at (2.5,-0.5) {$n=0$};
    \end{scope}

    \begin{scope}[xshift=6cm]
        \draw (0,0) grid (5, 5);
        \draw[thick, scale=5] (0, 0) grid (1, 1);

        \fillcell{2}{2}{dying}
        \fillcell{3}{2}{alive}
        \fillcell{2}{3}{alive}
        \fillcell{1}{3}{nascent}
        \fillcell{3}{1}{nascent}

        \node[anchor=center] at (2.5,-0.5) {$n=1$};
    \end{scope}

    \begin{scope}[xshift=12cm]
        \draw (0,0) grid (5, 5);
        \draw[thick, scale=5] (0, 0) grid (1, 1);

        \fillcell{3}{2}{dying}
        \fillcell{2}{3}{dying}
        \fillcell{1}{3}{alive}
        \fillcell{3}{1}{alive}
        \fillcell{1}{2}{nascent}
        \fillcell{2}{1}{nascent}
        \fillcell{3}{3}{nascent}

        \node[anchor=center] at (2.5,-0.5) {$n=2$};
    \end{scope}

    \begin{scope}[xshift=18cm]
        \draw (0,0) grid (5, 5);
        \draw[thick, scale=5] (0, 0) grid (1, 1);

        \fillcell{1}{3}{alive}
        \fillcell{3}{1}{alive}
        \fillcell{1}{2}{alive}
        \fillcell{2}{1}{alive}
        \fillcell{3}{3}{alive}
        \fillcell{4}{2}{nascent}
        \fillcell{2}{4}{nascent}

        \node[anchor=center] at (2.5,-0.5) {$n\geq3$};
    \end{scope}
\end{tikzpicture}

    \caption{Stabilisation de la configuration en quatre états}
    \label{fig:life-ex-3}
\end{figure}

\section{Modélisation}

Les cellules peuvent prendre uniquement deux états, vivante ou
morte. Pour cette raison, il est naturel de représenter une
cellule par un booléen, \texttt{true} pour une cellule
vivante et \texttt{false} pour une cellule morte.
Pour la représentation d'une grille de cellules, on utilisera donc
un tableau de tableaux de booléens.

Les tableaux seront de taille suffisamment grande pour éviter qu'une
structure n'arrive à la bordure. En effet, le comportement à la bordure
d'une grille n'est pas défini dans les règles du jeu de la vie puisqu'il
est censé se dérouler sur une grille infinie.

\section{Algorithmes}

Quatre algorithmes doivent être définis pour gérer la grille.

\begin{description}
    \item[Initialisation de la grille] Allocation d'une grille de taille
    donnée et initialisation de tous les booléens à \texttt{false}.
    \item[Détermination du nombre de voisins vivants] Reçoit une grille $G$
    et une position de cellule $(x, y)$.
    \begin{enumerate}
        \item Soit $total := 0$.
        \item Pour $i$ allant de $x - 1$ à $x + 1$~:
        \begin{enumerate}
            \item pour $j$ allant de $y - 1$ à $y + 1$, si
            $(i, j)$ est une position valide et
            $(i, j)$ est une cellule vivante de $G$ et
            $(i, j) \neq (x, y)$, alors $total := total + 1$.
        \end{enumerate}
        \item Renvoyer $total$.
    \end{enumerate}
    \item[Calcul de l'état suivant] Reçoit une grille $G$ et calcule
    la grille représentant l'état suivant du jeu.
    \begin{enumerate}
        \item Soit $G^\prime$ une grille de même taille que $G$.
        \item Pour $i$ allant de $0$ à la taille horizontale de $G$~:
        \begin{enumerate}
            \item pour $j$ allant de $0$ à la taille verticale de $G$~:
            \begin{enumerate}
                \item appeler l'algorithme \textbf{Détermination du nombre de voisins vivants}
                sur la grille $G$ et la case $(i, j)$ et stocker le résultat dans $voisins$~;
                \item si $voisins = 3$, $G^\prime(i, j) := \mathtt{true}$~;
                \item sinon, si $voisins = 2$, $G^\prime(i, j) := G(i, j)$~;
                \item sinon, $G^\prime(i, j) := \mathtt{false}$.
            \end{enumerate}
        \end{enumerate}
        \item Renvoyer $G^\prime$.
    \end{enumerate}
    \item[Affichage de la grille] Doit parcourir chaque cellule
    d'une grille donnée pour afficher à l'écran l'état de celle-ci. L'algorithme
    dépendra principalement du type d'affichage choisi (terminal, fenêtré).
\end{description}

Le programme principal se charge d'appeler l'algorithme 1, puis
d'appeler de manière répétée les algorithmes 3 et 4 pour afficher
les états suivants.

\section{Spécifications}

\subsection{Version initiale}

On choisira le langage C++, qui possède les structures requises dans
la section précédente, et est enseigné dans le cursus. Il n'y a pas de difficulté
algorithmique particulière qui justifie le choix d'un langage différent,
sachant que le choix d'un tel langage pourrait ralentir le développement.

L'affichage de la grille se fera dans le terminal. Un algorithme
naïf sera choisi pour le calcul de l'état suivant, se contenant de parcourir
chaque cellule et de calculer son état, sans recherche d'optimisation.

L'état initial sera choisi par l'utilisateur parmi une série de
préconfigurations codées \emph{en dur} dans le programme.

Le type \texttt{std::vector<std::vector<bool>\null>} de la librairie
standard C++ sera utilisé pour représenter les grilles car il permet
une allocation dynamique automatiquement gérée.

\subsection{Améliorations possibles}

L'affichage peut s'effectuer en fenêtré. Les bibliothèques
SDL~(en C) ou SFML~(en C++, orienté objet) peuvent être choisies
et seront utilisées dans l'algorithme d'affichage pour le dessin
des cellules, et dans le programme principal pour initialiser la fenêtre.

On peut laisser l'utilisateur choisir l'état initial en cliquant
sur les cellules. L'utilisateur pourra contrôler la génération grâce
à un bouton marche/arrêt, pas à pas, ou remise à zéro. On peut
également fournir une série de préconfigurations parmi lesquelles choisir
dans l'interface.

Enfin, un algorithme plus performant peut être utilisé, réduisant
les calculs inutiles. Par exemple, l'algorithme \emph{Hashlife,}
\cite{gof-hashlife}
utilisant des \emph{quadtrees} et une table de hachage peut améliorer
considérablement les performances (mais peut être difficile à implémenter).

\section{Organisation}

Pour la version initiale du programme, il y a quatre algorithmes à
mettre en place. Ces algorithmes reprennent des concepts
vus en cours. On pourra y consacrer 10 heures.

Pour l'affichage fenêtré, il faudra se renseigner sur l'A.P.I. des
bibliothèques utilisées, concevoir une interface (placement des boutons,
de la grille) et l'implémenter. On pourra prévoir 10 heures.

L'implémentation de l'algorithme \emph{Hashlife} requièrera des renseignements
sur l'utilisation des \emph{quadtrees} et des tables de hachages ainsi
que sur l'algorithme en lui-même. Prévoir 20 heures.

La rédaction du rapport s'effectuera en continu pendant la création du jeu.
La figure~\ref{fig:life-gantt} présente un diagramme de Gantt résumant
la répartition du travail.

\begin{figure}[h!]
    \centering
    \begin{gantt}{5}{8}
    \begin{ganttitle}
        \numtitle{5}{5}{40}{1}
    \end{ganttitle}

    \ganttbar{Base}{0}{2}
    \ganttbar{Fenêtrage}{2}{2}
    \ganttbar{Hashlife}{4}{4}
    \ganttbar{Rapport}{0}{8}
\end{gantt}

    \caption{Développement du jeu de la vie sur 40 heures}
    \label{fig:life-gantt}
\end{figure}
